La población ocupada\footnote{Personas de 15 años o más, que durante la semana de referencia hayan realizado durante una hora o un día, alguna actividad económica, trabajando en el período de referencia por un sueldo o salario en metálico o especie o ausentes temporalmente de su trabajo} no asalariada que trabaja por cuenta propia, no posee una relación contractual, ni goza de los beneficios de aguinaldo, bono 14, horas extras, etc., además de no tener acceso a seguridad social.

 Para 2000, casi   la tercera parte de los ocupados trabajaba de forma independiente. Esta proporción se redujo en el 2014 a 26.4\%.