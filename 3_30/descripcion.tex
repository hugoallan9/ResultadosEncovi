Contar con saneamiento mejorado puede ayudar a reducir la prevalencia de  enfermedades infecciosas  en menores, así como también los niveles de desnutrición crónica, aguda y global. 

El 21.4\% de los hogares en Alta Verapaz y el 30.1\% en Totonicapán tiene acceso a saneamiento mejorado\footnote{Incluye inodoro conectado a red de drenaje, inodoro conectado a fosa séptica y excusado lavable}, mientras que en los departamentos de Guatemala y Sacatepéquez, casi el 90\% de los hogares cuenta con servicios de saneamiento mejorados.