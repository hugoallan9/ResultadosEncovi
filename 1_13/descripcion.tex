Entre 2000 y 2006, la brecha de pobreza\footnote{Muestra la distancia promedio de la población pobre respecto de  la línea de pobreza total. Permite estimar la transferencia necesaria para  que los hogares de escasos recursos salgan de la  pobreza. \\\\ Se calcula mediante la fórmula: 
\[ FGT(1) = \frac {1} {N} \sum_{i=1}^H \left(\frac {z-x_i} {z}\right),   \] donde $x_i$ es el consumo de la persona $i$, $H$ es  la cantidad de casos que reportan un consumo menor  o igual a la línea de pobreza total, $N$ es el la población tota y $z$ es el valor de la línea de pobreza total.   } se redujo en 3.2 puntos porcentuales. No obstante, entre 2006 y 2014, la distancia de la población pobre a la línea de pobreza total aumentó a 22.0\%, casi el mismo valor  que para el año 2000.