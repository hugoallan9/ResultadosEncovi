Al aumentar el parámetro de aversión a la desigualdad, donde se pone mayor énfasis al extremo inferior de la distribución  ($\varepsilon = \mbox{2}$), los resultados que se venían observando se modifican. El departamento de San Marcos (0.83), sigue mostrando el mayor nivel desigualdad; no obstante, se observa cómo aumenta el índice de Atkinson para otros departamentos, como Totonicapán  que con $\varepsilon = \mbox{1}$ era uno de los menos desiguales y con $\varepsilon = \mbox{2}$ es el segundo más desigual (0.77).  \\\\ Los departamentos de Sololá (0.45), Escuintla (0.45) y El Progreso (0.48), muestran los niveles de desigualdad más bajos con el índice de Atkinson con  $\varepsilon = \mbox{2}$.