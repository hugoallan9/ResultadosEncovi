La brecha de pobreza, muestra la distancia promedio a la que se encuentra la población en pobreza a la línea de pobreza total. Este indicador permite hacer una estimación del monto que sería necesario transferir a los hogares de escasos recursos, para poder salir de la pobreza.   Entre 2000 y 2006, la brecha de pobreza se redujo en 3.2 puntos porcentuales. No obstante, entre 2006 y 2014, la distancia de la población a la línea de pobreza total aumentó 22.0\%, casi el misma distancia observada que para el año 2000.