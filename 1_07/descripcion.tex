\textollamada{En el mapa, los valores positivos significan un aumento de pobreza, y los negativos una disminución. } De los veintidós departamentos del país dieciocho aumentaron su porcentaje de pobreza entre 2006 y 2014 y cuatro lo disminuyeron. 

Se puede observar que el mayor aumento en la incidencia de pobreza  en este período, se dio en el departamento de Guatemala con un aumento de 17 puntos porcentuales; le siguen los departamentos de Jutiapa, Quetzaltenango, Escuintla, El Progreso y Chiquimula, con un aumento de más de 10 puntos porcentuales. Los departamentos que muestran una reducción en la incidencia de  pobreza son Quiché, San Marcos, Baja Verapaz y Santa Rosa. 