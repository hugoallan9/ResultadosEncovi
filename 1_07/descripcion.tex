Los resultados positivos de la comparación entre 2006 y 2014, significan un aumento de la pobreza en este período. Se puede observar que el mayor aumento entre estos dos años, se dio en el departamento de Guatemala, con un aumento de 17 puntos porcentuales en la pobreza. Le siguen los departamentos de Jutiapa, Quetzaltenango, Escuintla, El Progreso y Chiquimula, con un aumento de más de 10 puntos porcentuales. Los departamentos que muestran una reducción de la pobreza son Quiché, San Marcos, Baja Verapaz y Santa Rosa. 