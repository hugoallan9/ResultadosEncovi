Cuando $\varepsilon = \mbox{2}$, el parámetro de aversión a la desigualdad aumenta y se otorga mayor peso al extremo inferior de la distribución\footnote{El índice de Atkinson con $\varepsilon = \mbox{2} $ se calcula con la fórmula: 
	\[ A = 1 - \frac{N}{\mu}\left( \sum_{i=1}^{N}\frac{1}{x_i} \right)^{-1}, \] donde $N$ es la población total, $\mu$ es la media del ingreso y $x_i$ es el ingreso de la persona $i$.}. \\\\ Entre 2006 y 2011, el índice de Atkinson aumentó de 0.72 a 0.77, respectivamente, y de 2011 a 2014, se redujo a 0.71.  
