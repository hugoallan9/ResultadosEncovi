La pobreza extrema es mayor en la población indígena. Para el año 2000, más de la cuarta parte de la población indígena se encontraba en pobreza extrema, se puede observar que entre 2000 y 2006 el nivel de pobreza se mantuvo, y aumentó a casi 40\% en 2014. Para la población no indígena, la pobreza extrema aumentó de cinco puntos porcentuales, de 7.8\% a 12.8\% entre 2000 y 2014.