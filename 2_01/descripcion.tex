Según los resultados de la Encovi, para el año 2000 el coeficiente de Gini\footnote{El coeficiente de Gini permite cuantificar la distancia de la distribución a la perfecta igualdad. Su valor varía entre 0 y 1, mientras más cerca se encuentre el valor del 1, mayor será la desigualdad. \\\\ 
	Se calcula con la fórmula: 
	\[ G= \sum_{i=1}^{n}\sum_{j=1}^{n}\frac{\mid x_i - x_j \mid}{2n^2\mu} \]} era 0.60. Entre 2006 y 2011, la desigualdad aumentó ligeramente de 0.56 a 0.57, respectivamente, mientras que de 2011 a 2014, se observó una reducción de la desigualdad a 0.53, por debajo de lo observado para el año 2006.