
\INEchaptercarta{Objetivos de Desarrollo del Milenio}{}

%
%\cajota{Indicadores de los Objetivos de Desarrollo al Milenio que se monitorean con la ENCOVI}{ }{}{ }{\small
%$\ $\\[-4cm]
%\ra{1.5}$\ $\\
%\begin{tabular}{cm{4cm}m{5cm}m{5cm}}\hline
%\rowcolor{color2!15!white} &  &    &    \\[-5.9mm]
%&Descripción &  Meta     &   Indicadores     \\
%\hline
%\%\rowcolor{white}   &    &     \\[-6mm]
%\multirow{5}{*}[-2cm]{\begin{sideways} {\large Objetivo 1} \end{sideways}}&\multirow{5}{4cm}[-3cm]{Erradicar la pobreza extrema y el hambre}&\multirow{2}{5cm}[-1cm]{Meta 1A: Reducir a la mitad entre 1990 y 2015, el porcentaje de personas cuyos ingresos sean inferiores a 1 dólar por día}&Proporción de la población que se encuentra por debajo de la línea de pobreza extrema\\
%&&&Proporción del consumo nacional que corresponde al quintil más pobre de la población\\
%&&&\\
%&&\multirow{2}{5cm}[-0.5cm]{Meta 1B: Lograr empleo pleno y productivo, y trabajo decente para todos, incluyendo mujeres y jóvenes}&Relación empleo - población\\
%&&&Proporción de la población ocupada que trabaja por cuenta propia o en una empresa familiar\\
%&&\\
%\multirow{1}{*}[0.4cm]{\begin{sideways} {\large Objetivo 2} \end{sideways}} & Lograr la enseñanza primaria universal &Meta 2A: Asegurar que, para el año 2015, los niños y niñas de todo el mundo puedan terminar un ciclo completo de enseñanza primaria y &Tasa de alfabetización de las personas de 15 a 24 años, mujeres y hombres\\
%&&&\\
%\multirow{1}{*}[0.1cm]{\begin{sideways} {\large Objetivo 3} \end{sideways}} & Promover la igualdad de género y el empoderamiento de la mujer&Meta 3A: Eliminar las desigualdades entre los sexos en la enseñanza primaria y secundaria, preferiblemente para el año 2005, y en todos los niveles de la enseñanza para el año 2015&Proporción de mujeres entre los empleados remunerados en el sector no agrícola\\
%&&&\\
%
%\multirow{2}{*}[0.1cm]{\begin{sideways} {\large Objetivo 5} \end{sideways}} & \multirow{2}{4cm}[0cm]{Mejorar la salud materna}& \multirow{2}{5cm}[0cm]{Meta 5A: Reducir, entre 1990 y 2015, la mortalidad materna en tres cuartas partes} &Proporción de partos con asistencia de personal sanitario especializado\\
%
%&&&Lugar de parto para los nacimientos\\
%&&&\\
%\multirow{1}{*}[0.1cm]{\begin{sideways} {\large Objetivo 7} \end{sideways}} & Garantizar la sostenibilidad del Medio ambiente&Meta 7C: Reducir a la mitad, para el año 2015, el porcentaje de personas sin acceso sostenible al agua potable y a servicios básicos de saneamiento&Proporción de la población con acceso a fuentes mejoradas de abastecimiento de agua potable\\
%&&&Proporción de la población con acceso a servicios de saneamiento mejorados\\
%\hline
%&    &      \\[-0.05cm]
%\end{tabular}
%}{}


\cajita{Población con gastos inferiores a la línea de pobreza extrema}{ 
Este indicador muestra el porcentaje de población que no logra cubrir el costo del consumo mínimo de alimentos. 

Se puede observar que en este período, la tendencia es a una reducción en la proporción de la población que se encuentra por debajo de la línea de pobreza extrema nacional, ya que entre 2000 y 2014, este se redujo de 15.7\% a \%. 
}{Proporción de la población que se encuentra por debajo de la línea de pobreza extrema}{República de Guatemala, serie histórica por Encovi, en porcentaje}{\ \\[0mm]\begin{tikzpicture}[x=1pt,y=1pt]  % Created by tikzDevice version 0.8.1 on 2015-11-05 13:53:57
% !TEX encoding = UTF-8 Unicode
\definecolor{fillColor}{RGB}{255,255,255}
\path[use as bounding box,fill=fillColor,fill opacity=0.00] (0,0) rectangle (289.08,198.74);
\begin{scope}
\path[clip] (  0.00,  0.00) rectangle (289.08,198.74);

\path[] (  0.00,  0.00) rectangle (289.08,198.74);
\end{scope}
\begin{scope}
\path[clip] (  0.00,  0.00) rectangle (289.08,198.74);

\path[] (  1.64, 17.78) rectangle (280.54,191.48);

\path[] (  1.64, 50.76) --
	(280.54, 50.76);

\path[] (  1.64,100.94) --
	(280.54,100.94);

\path[] (  1.64,151.11) --
	(280.54,151.11);

\path[] (  1.64, 25.67) --
	(280.54, 25.67);

\path[] (  1.64, 75.85) --
	(280.54, 75.85);

\path[] (  1.64,126.02) --
	(280.54,126.02);

\path[] (  1.64,176.20) --
	(280.54,176.20);

\path[] ( 41.49, 17.78) --
	( 41.49,191.48);

\path[] (107.89, 17.78) --
	(107.89,191.48);

\path[] (174.30, 17.78) --
	(174.30,191.48);

\path[] (240.70, 17.78) --
	(240.70,191.48);
\definecolor{drawColor}{RGB}{0,0,255}

\path[draw=drawColor,line width= 1.7pt,line join=round] ( 41.49,183.59) --
	(107.89,178.21) --
	(174.30,159.14) --
	(240.70, 25.67);
\definecolor{drawColor}{RGB}{0,0,0}

\node[text=drawColor,anchor=base,inner sep=0pt, outer sep=0pt, scale=  1.01] at ( 41.49,187.54) {15.7};

\node[text=drawColor,anchor=base west,inner sep=0pt, outer sep=0pt, scale=  1.01] at (107.89,182.17) {15.2};

\node[text=drawColor,anchor=base west,inner sep=0pt, outer sep=0pt, scale=  1.01] at (174.30,163.10) {13.3};

\node[text=drawColor,anchor=base,inner sep=0pt, outer sep=0pt, scale=  1.01] at (240.70, 13.80) {0.0};

\path[draw=drawColor,line width= 0.1pt,line join=round] (  1.64, 25.67) -- (280.54, 25.67);

\path[] (  1.64, 17.78) rectangle (280.54,191.48);
\end{scope}
\begin{scope}
\path[clip] (  0.00,  0.00) rectangle (289.08,198.74);

\path[] (  1.64, 17.78) --
	(  1.64,191.48);
\end{scope}
\begin{scope}
\path[clip] (  0.00,  0.00) rectangle (289.08,198.74);

\path[] (  0.00, 25.67) --
	(  1.64, 25.67);

\path[] (  0.00, 75.85) --
	(  1.64, 75.85);

\path[] (  0.00,126.02) --
	(  1.64,126.02);

\path[] (  0.00,176.20) --
	(  1.64,176.20);
\end{scope}
\begin{scope}
\path[clip] (  0.00,  0.00) rectangle (289.08,198.74);

\path[] (  1.64, 17.78) --
	(280.54, 17.78);
\end{scope}
\begin{scope}
\path[clip] (  0.00,  0.00) rectangle (289.08,198.74);

\path[] ( 41.49, 13.51) --
	( 41.49, 17.78);

\path[] (107.89, 13.51) --
	(107.89, 17.78);

\path[] (174.30, 13.51) --
	(174.30, 17.78);

\path[] (240.70, 13.51) --
	(240.70, 17.78);
\end{scope}
\begin{scope}
\path[clip] (  0.00,  0.00) rectangle (289.08,198.74);
\definecolor{drawColor}{RGB}{0,0,0}

\node[text=drawColor,anchor=base,inner sep=0pt, outer sep=0pt, scale=  1.00] at ( 41.49,  2.85) {2000};

\node[text=drawColor,anchor=base,inner sep=0pt, outer sep=0pt, scale=  1.00] at (107.89,  2.85) {2006};

\node[text=drawColor,anchor=base,inner sep=0pt, outer sep=0pt, scale=  1.00] at (174.30,  2.85) {2011};

\node[text=drawColor,anchor=base,inner sep=0pt, outer sep=0pt, scale=  1.00] at (240.70,  2.85) {2014};
\end{scope}
  \end{tikzpicture}}{Instituto Nacional de estadística}{}
\cajita{Población con gastos inferiores a la línea de pobreza extrema por área de residencia}{ 0}{Proporción de la población que se encuentra por debajo de la línea de pobreza extrema por área de residencia}{República de Guatemala, Encovi 2014, en porcentaje}{\ \\[0mm]\begin{tikzpicture}[x=1pt,y=1pt]  \input{graficas/odm/25_02.tex}  \end{tikzpicture}}{Instituto Nacional de estadística}{}
\cajota{Población con gastos inferiores a la línea de pobreza extrema en los departamentos}{ 0}{Proporción de la población que se encuentra por debajo de la línea de pobreza extrema por departamento}{República de Guatemala, Encovi 2014, en porcentaje}{\ \\[0mm]\begin{tikzpicture}[x=1pt,y=1pt]  \input{graficas/odm/25_02.tex}  \end{tikzpicture}}{Instituto Nacional de Estadística}{}
\cajita{Consumo nacional de la quinta parte de la población más pobre}{ La participación de cada quintil en el consumo nacional, es un indicador que permite evidenciar las  desigualdades entre los distintos estratos de la población. Para el 2014, el 20\% más  pobre de la población captaba el \% del consumo nacional. 
Se puede observar que en este período, la tendencia del indicador es a un aumento, ya que entre 2000 y 2014, pasó de 5.2\% a \%. 
}{Proporción del consumo nacional que corresponde al quintil más pobre de la población}{República de Guatemala, serie histórica por Encovi, en porcentaje}{\ \\[0mm]\begin{tikzpicture}[x=1pt,y=1pt]  % Created by tikzDevice version 0.8.1 on 2015-11-05 13:53:59
% !TEX encoding = UTF-8 Unicode
\definecolor{fillColor}{RGB}{255,255,255}
\path[use as bounding box,fill=fillColor,fill opacity=0.00] (0,0) rectangle (289.08,198.74);
\begin{scope}
\path[clip] (  0.00,  0.00) rectangle (289.08,198.74);

\path[] (  0.00,  0.00) rectangle (289.08,198.74);
\end{scope}
\begin{scope}
\path[clip] (  0.00,  0.00) rectangle (289.08,198.74);

\path[] ( -2.73, 17.78) rectangle (280.54,191.48);

\path[] (  0.00, 47.68) --
	(280.54, 47.68);

\path[] (  0.00, 91.71) --
	(280.54, 91.71);

\path[] (  0.00,135.73) --
	(280.54,135.73);

\path[] (  0.00,179.76) --
	(280.54,179.76);

\path[] (  0.00, 25.67) --
	(280.54, 25.67);

\path[] (  0.00, 69.70) --
	(280.54, 69.70);

\path[] (  0.00,113.72) --
	(280.54,113.72);

\path[] (  0.00,157.75) --
	(280.54,157.75);

\path[] ( 37.73, 17.78) --
	( 37.73,191.48);

\path[] (105.18, 17.78) --
	(105.18,191.48);

\path[] (172.63, 17.78) --
	(172.63,191.48);

\path[] (240.08, 17.78) --
	(240.08,191.48);
\definecolor{drawColor}{RGB}{0,0,255}

\path[draw=drawColor,line width= 1.7pt,line join=round] ( 37.73,139.48) --
	(105.18,149.26) --
	(172.63,183.59) --
	(240.08, 25.67);
\definecolor{drawColor}{RGB}{0,0,0}

\node[text=drawColor,anchor=base,inner sep=0pt, outer sep=0pt, scale=  1.01] at ( 37.73,127.61) {5.2};

\node[text=drawColor,anchor=base east,inner sep=0pt, outer sep=0pt, scale=  1.01] at (102.95,149.26) {5.6};

\node[text=drawColor,anchor=base,inner sep=0pt, outer sep=0pt, scale=  1.01] at (172.63,187.54) {7.2};

\node[text=drawColor,anchor=base,inner sep=0pt, outer sep=0pt, scale=  1.01] at (240.08, 13.80) {0.0};

\path[draw=drawColor,line width= 0.1pt,line join=round] (  0.00, 25.67) -- (280.54, 25.67);

\path[] ( -2.73, 17.78) rectangle (280.54,191.48);
\end{scope}
\begin{scope}
\path[clip] (  0.00,  0.00) rectangle (289.08,198.74);

\path[] (  0.00, 17.78) --
	(280.54, 17.78);
\end{scope}
\begin{scope}
\path[clip] (  0.00,  0.00) rectangle (289.08,198.74);

\path[] ( 37.73, 13.51) --
	( 37.73, 17.78);

\path[] (105.18, 13.51) --
	(105.18, 17.78);

\path[] (172.63, 13.51) --
	(172.63, 17.78);

\path[] (240.08, 13.51) --
	(240.08, 17.78);
\end{scope}
\begin{scope}
\path[clip] (  0.00,  0.00) rectangle (289.08,198.74);
\definecolor{drawColor}{RGB}{0,0,0}

\node[text=drawColor,anchor=base,inner sep=0pt, outer sep=0pt, scale=  1.00] at ( 37.73,  2.85) {2000};

\node[text=drawColor,anchor=base,inner sep=0pt, outer sep=0pt, scale=  1.00] at (105.18,  2.85) {2006};

\node[text=drawColor,anchor=base,inner sep=0pt, outer sep=0pt, scale=  1.00] at (172.63,  2.85) {2011};

\node[text=drawColor,anchor=base,inner sep=0pt, outer sep=0pt, scale=  1.00] at (240.08,  2.85) {2014};
\end{scope}
  \end{tikzpicture}}{Instituto Nacional de Estadística}{}
\cajita{Consumo nacional del quintil más pobre por área de residencia}{ 0}{Proporción del consumo nacional que corresponde al quintil más pobre de la población por área de residencia}{República de Guatemala, Encovi 2014, en porcentaje}{\ \\[0mm]\begin{tikzpicture}[x=1pt,y=1pt]  \input{graficas/odm/25_05.tex}  \end{tikzpicture}}{Instituto Nacional de Estadística}{}
\cajota{Consumo nacional del quintil más pobre en los departamentos}{ 0}{Proporción del consumo nacional que corresponde al quintil más pobre de la población por departamento}{República de Guatemala, Encovi 2014, en porcentaje}{\ \\[0mm]\begin{tikzpicture}[x=1pt,y=1pt]  \input{graficas/odm/25_02.tex}  \end{tikzpicture}}{Instituto Nacional de Estadística}{}
\cajita{Empleo pleno y productivo}{ Este indicador muestra la capacidad de la economía de generar empleo. Es un indicador cuantitativo y no refleja la calidad del empleo a lo largo del tiempo, ya que se considera al total de la población ocupada. 
Entre 2000 y 2014 no se observa una tendencia específica en este indicador, ya que se ha mantenido entre 62.7\% y 64.9\%. No existe una relación empleo-población óptima.}{Relación entre empleo y población}{República de Guatemala, serie histórica por Encovi, en porcentaje}{\ \\[0mm]\begin{tikzpicture}[x=1pt,y=1pt]  % Created by tikzDevice version 0.8.1 on 2015-11-05 13:54:02
% !TEX encoding = UTF-8 Unicode
\definecolor{fillColor}{RGB}{255,255,255}
\path[use as bounding box,fill=fillColor,fill opacity=0.00] (0,0) rectangle (289.08,198.74);
\begin{scope}
\path[clip] (  0.00,  0.00) rectangle (289.08,198.74);

\path[] (  0.00,  0.00) rectangle (289.08,198.74);
\end{scope}
\begin{scope}
\path[clip] (  0.00,  0.00) rectangle (289.08,198.74);

\path[] (  1.64, 17.78) rectangle (280.54,191.48);

\path[] (  1.64, 50.01) --
	(280.54, 50.01);

\path[] (  1.64, 98.68) --
	(280.54, 98.68);

\path[] (  1.64,147.35) --
	(280.54,147.35);

\path[] (  1.64, 25.67) --
	(280.54, 25.67);

\path[] (  1.64, 74.34) --
	(280.54, 74.34);

\path[] (  1.64,123.01) --
	(280.54,123.01);

\path[] (  1.64,171.68) --
	(280.54,171.68);

\path[] ( 41.49, 17.78) --
	( 41.49,191.48);

\path[] (107.89, 17.78) --
	(107.89,191.48);

\path[] (174.30, 17.78) --
	(174.30,191.48);

\path[] (240.70, 17.78) --
	(240.70,191.48);
\definecolor{drawColor}{RGB}{0,0,255}

\path[draw=drawColor,line width= 1.7pt,line join=round] ( 41.49,178.32) --
	(107.89,183.59) --
	(174.30,178.53) --
	(240.70,173.69);
\definecolor{drawColor}{RGB}{0,0,0}

\node[text=drawColor,anchor=base,inner sep=0pt, outer sep=0pt, scale=  1.01] at ( 41.49,166.44) {62.7};

\node[text=drawColor,anchor=base,inner sep=0pt, outer sep=0pt, scale=  1.01] at (107.89,187.54) {64.9};

\node[text=drawColor,anchor=base west,inner sep=0pt, outer sep=0pt, scale=  1.01] at (174.30,182.48) {62.8};

\node[text=drawColor,anchor=base,inner sep=0pt, outer sep=0pt, scale=  1.01] at (240.70,161.82) {60.8};

\path[draw=drawColor,line width= 0.1pt,line join=round] (  1.64, 25.67) -- (280.54, 25.67);

\path[] (  1.64, 17.78) rectangle (280.54,191.48);
\end{scope}
\begin{scope}
\path[clip] (  0.00,  0.00) rectangle (289.08,198.74);

\path[] (  1.64, 17.78) --
	(  1.64,191.48);
\end{scope}
\begin{scope}
\path[clip] (  0.00,  0.00) rectangle (289.08,198.74);

\path[] (  0.00, 25.67) --
	(  1.64, 25.67);

\path[] (  0.00, 74.34) --
	(  1.64, 74.34);

\path[] (  0.00,123.01) --
	(  1.64,123.01);

\path[] (  0.00,171.68) --
	(  1.64,171.68);
\end{scope}
\begin{scope}
\path[clip] (  0.00,  0.00) rectangle (289.08,198.74);

\path[] (  1.64, 17.78) --
	(280.54, 17.78);
\end{scope}
\begin{scope}
\path[clip] (  0.00,  0.00) rectangle (289.08,198.74);

\path[] ( 41.49, 13.51) --
	( 41.49, 17.78);

\path[] (107.89, 13.51) --
	(107.89, 17.78);

\path[] (174.30, 13.51) --
	(174.30, 17.78);

\path[] (240.70, 13.51) --
	(240.70, 17.78);
\end{scope}
\begin{scope}
\path[clip] (  0.00,  0.00) rectangle (289.08,198.74);
\definecolor{drawColor}{RGB}{0,0,0}

\node[text=drawColor,anchor=base,inner sep=0pt, outer sep=0pt, scale=  1.00] at ( 41.49,  2.85) {2000};

\node[text=drawColor,anchor=base,inner sep=0pt, outer sep=0pt, scale=  1.00] at (107.89,  2.85) {2006};

\node[text=drawColor,anchor=base,inner sep=0pt, outer sep=0pt, scale=  1.00] at (174.30,  2.85) {2011};

\node[text=drawColor,anchor=base,inner sep=0pt, outer sep=0pt, scale=  1.00] at (240.70,  2.85) {2014};
\end{scope}
  \end{tikzpicture}}{Instituto Nacional de Estadística}{}
\cajita{Empleo pleno y productivo por área de residencia}{ 0}{Relación entre empleo y población por área de residencia}{República de Guatemala, Encovi 2014, en porcentaje}{\ \\[0mm]\begin{tikzpicture}[x=1pt,y=1pt]  \input{graficas/odm/25_08.tex}  \end{tikzpicture}}{Instituto Nacional de Estadística}{}
\cajota{Empleo pleno y productivo en los departamentos}{ 0}{Relación entre empleo y población por departamento}{República de Guatemala, Encovi 2014, en porcentaje}{\ \\[0mm]\begin{tikzpicture}[x=1pt,y=1pt]  \input{graficas/odm/25_02.tex}  \end{tikzpicture}}{Instituto Nacional de Estadística}{}
\cajita{Población ocupada no asalariada}{ Este indicador permite hacer una medición del empleo independiente con rasgos de vulnerabilidad, ya que los trabajadores familiares, se encuentran en una categoría vulnerable ya que no tienen relación contractual, ni beneficios, ni seguridad social, etc.

La tendencia del indicador entre 2000 y 2014, es a una reducción en la proporción de la población ocupada que trabaja como cuenta propia, de 31.2\% a x\%.
}{Proporción de la población ocupada que trabaja por cuenta propia o en una empresa familiar}{República de Guatemala, serie histórica por Encovi, en porcentaje}{\ \\[0mm]\begin{tikzpicture}[x=1pt,y=1pt]  % Created by tikzDevice version 0.8.1 on 2015-11-05 13:54:04
% !TEX encoding = UTF-8 Unicode
\definecolor{fillColor}{RGB}{255,255,255}
\path[use as bounding box,fill=fillColor,fill opacity=0.00] (0,0) rectangle (289.08,198.74);
\begin{scope}
\path[clip] (  0.00,  0.00) rectangle (289.08,198.74);

\path[] (  0.00,  0.00) rectangle (289.08,198.74);
\end{scope}
\begin{scope}
\path[clip] (  0.00,  0.00) rectangle (289.08,198.74);

\path[] (  1.64, 17.78) rectangle (280.54,191.48);

\path[] (  1.64, 50.96) --
	(280.54, 50.96);

\path[] (  1.64,101.53) --
	(280.54,101.53);

\path[] (  1.64,152.10) --
	(280.54,152.10);

\path[] (  1.64, 25.67) --
	(280.54, 25.67);

\path[] (  1.64, 76.24) --
	(280.54, 76.24);

\path[] (  1.64,126.82) --
	(280.54,126.82);

\path[] (  1.64,177.39) --
	(280.54,177.39);

\path[] ( 41.49, 17.78) --
	( 41.49,191.48);

\path[] (107.89, 17.78) --
	(107.89,191.48);

\path[] (174.30, 17.78) --
	(174.30,191.48);

\path[] (240.70, 17.78) --
	(240.70,191.48);
\definecolor{drawColor}{RGB}{0,0,255}

\path[draw=drawColor,line width= 1.7pt,line join=round] ( 41.49,183.59) --
	(107.89,182.67) --
	(174.30,164.61) --
	(240.70,159.12);
\definecolor{drawColor}{RGB}{0,0,0}

\node[text=drawColor,anchor=base,inner sep=0pt, outer sep=0pt, scale=  1.01] at ( 41.49,187.54) {31.2};

\node[text=drawColor,anchor=base west,inner sep=0pt, outer sep=0pt, scale=  1.01] at (107.89,186.62) {31.0};

\node[text=drawColor,anchor=base west,inner sep=0pt, outer sep=0pt, scale=  1.01] at (174.30,168.56) {27.5};

\node[text=drawColor,anchor=base,inner sep=0pt, outer sep=0pt, scale=  1.01] at (240.70,147.25) {26.4};

\path[draw=drawColor,line width= 0.1pt,line join=round] (  1.64, 25.67) -- (280.54, 25.67);

\path[] (  1.64, 17.78) rectangle (280.54,191.48);
\end{scope}
\begin{scope}
\path[clip] (  0.00,  0.00) rectangle (289.08,198.74);

\path[] (  1.64, 17.78) --
	(  1.64,191.48);
\end{scope}
\begin{scope}
\path[clip] (  0.00,  0.00) rectangle (289.08,198.74);

\path[] (  0.00, 25.67) --
	(  1.64, 25.67);

\path[] (  0.00, 76.24) --
	(  1.64, 76.24);

\path[] (  0.00,126.82) --
	(  1.64,126.82);

\path[] (  0.00,177.39) --
	(  1.64,177.39);
\end{scope}
\begin{scope}
\path[clip] (  0.00,  0.00) rectangle (289.08,198.74);

\path[] (  1.64, 17.78) --
	(280.54, 17.78);
\end{scope}
\begin{scope}
\path[clip] (  0.00,  0.00) rectangle (289.08,198.74);

\path[] ( 41.49, 13.51) --
	( 41.49, 17.78);

\path[] (107.89, 13.51) --
	(107.89, 17.78);

\path[] (174.30, 13.51) --
	(174.30, 17.78);

\path[] (240.70, 13.51) --
	(240.70, 17.78);
\end{scope}
\begin{scope}
\path[clip] (  0.00,  0.00) rectangle (289.08,198.74);
\definecolor{drawColor}{RGB}{0,0,0}

\node[text=drawColor,anchor=base,inner sep=0pt, outer sep=0pt, scale=  1.00] at ( 41.49,  2.85) {2000};

\node[text=drawColor,anchor=base,inner sep=0pt, outer sep=0pt, scale=  1.00] at (107.89,  2.85) {2006};

\node[text=drawColor,anchor=base,inner sep=0pt, outer sep=0pt, scale=  1.00] at (174.30,  2.85) {2011};

\node[text=drawColor,anchor=base,inner sep=0pt, outer sep=0pt, scale=  1.00] at (240.70,  2.85) {2014};
\end{scope}
  \end{tikzpicture}}{Instituto Nacional de Estadística}{}
\cajita{Población ocupada no asalariada por área de residencia}{ 0}{Proporción de la población ocupada que trabaja por cuenta propia o en una empresa familiar por área de residencia}{República de Guatemala, Encovi 2014, en porcentaje}{\ \\[0mm]\begin{tikzpicture}[x=1pt,y=1pt]  \input{graficas/odm/25_11.tex}  \end{tikzpicture}}{Instituto Nacional de Estadística}{}
\cajota{Población ocupada no asalariada en los departamentos}{ 0}{Proporción de la población ocupada que trabaja por cuenta propia o en una empresa familiar por departamento}{República de Guatemala, Encovi 2014, en porcentaje}{\ \\[0mm]\begin{tikzpicture}[x=1pt,y=1pt]  \input{graficas/odm/25_11.tex}  \end{tikzpicture}}{Instituto Nacional de Estadística}{}
\cajita{Alfabetismo en jóvenes}{ La tasa de alfabetización de los jóvenes entre 15 y 24 años, permite registrar en generaciones de mayor edad, la asistencia a la educación primaria. 

En la gráfica se puede observar que la tendencia en este período, es a un aumento en la tasa de alfabetización de jóvenes, de 81.7\% en 2000 a \% en 2014.
}{Tasa de alfabetismo en población de 15 a 24 años}{República de Guatemala, serie histórica por Encovi, en porcentaje}{\ \\[0mm]\begin{tikzpicture}[x=1pt,y=1pt]  % Created by tikzDevice version 0.8.1 on 2015-11-05 13:54:07
% !TEX encoding = UTF-8 Unicode
\definecolor{fillColor}{RGB}{255,255,255}
\path[use as bounding box,fill=fillColor,fill opacity=0.00] (0,0) rectangle (289.08,198.74);
\begin{scope}
\path[clip] (  0.00,  0.00) rectangle (289.08,198.74);

\path[] (  0.00,  0.00) rectangle (289.08,198.74);
\end{scope}
\begin{scope}
\path[clip] (  0.00,  0.00) rectangle (289.08,198.74);

\path[] (  1.64, 17.78) rectangle (280.54,191.48);

\path[] (  1.64, 46.82) --
	(280.54, 46.82);

\path[] (  1.64, 89.11) --
	(280.54, 89.11);

\path[] (  1.64,131.41) --
	(280.54,131.41);

\path[] (  1.64,173.70) --
	(280.54,173.70);

\path[] (  1.64, 25.67) --
	(280.54, 25.67);

\path[] (  1.64, 67.96) --
	(280.54, 67.96);

\path[] (  1.64,110.26) --
	(280.54,110.26);

\path[] (  1.64,152.55) --
	(280.54,152.55);

\path[] ( 41.49, 17.78) --
	( 41.49,191.48);

\path[] (107.89, 17.78) --
	(107.89,191.48);

\path[] (174.30, 17.78) --
	(174.30,191.48);

\path[] (240.70, 17.78) --
	(240.70,191.48);
\definecolor{drawColor}{RGB}{0,0,255}

\path[draw=drawColor,line width= 1.7pt,line join=round] ( 41.49,163.89) --
	(107.89,174.21) --
	(174.30,179.79) --
	(240.70,183.59);
\definecolor{drawColor}{RGB}{0,0,0}

\node[text=drawColor,anchor=base,inner sep=0pt, outer sep=0pt, scale=  1.01] at ( 41.49,152.02) {81.7};

\node[text=drawColor,anchor=base east,inner sep=0pt, outer sep=0pt, scale=  1.01] at (104.78,174.21) {87.8};

\node[text=drawColor,anchor=base east,inner sep=0pt, outer sep=0pt, scale=  1.01] at (171.18,179.79) {91.1};

\node[text=drawColor,anchor=base,inner sep=0pt, outer sep=0pt, scale=  1.01] at (240.70,187.54) {93.3};

\path[draw=drawColor,line width= 0.1pt,line join=round] (  1.64, 25.67) -- (280.54, 25.67);

\path[] (  1.64, 17.78) rectangle (280.54,191.48);
\end{scope}
\begin{scope}
\path[clip] (  0.00,  0.00) rectangle (289.08,198.74);

\path[] (  1.64, 17.78) --
	(  1.64,191.48);
\end{scope}
\begin{scope}
\path[clip] (  0.00,  0.00) rectangle (289.08,198.74);

\path[] (  0.00, 25.67) --
	(  1.64, 25.67);

\path[] (  0.00, 67.96) --
	(  1.64, 67.96);

\path[] (  0.00,110.26) --
	(  1.64,110.26);

\path[] (  0.00,152.55) --
	(  1.64,152.55);
\end{scope}
\begin{scope}
\path[clip] (  0.00,  0.00) rectangle (289.08,198.74);

\path[] (  1.64, 17.78) --
	(280.54, 17.78);
\end{scope}
\begin{scope}
\path[clip] (  0.00,  0.00) rectangle (289.08,198.74);

\path[] ( 41.49, 13.51) --
	( 41.49, 17.78);

\path[] (107.89, 13.51) --
	(107.89, 17.78);

\path[] (174.30, 13.51) --
	(174.30, 17.78);

\path[] (240.70, 13.51) --
	(240.70, 17.78);
\end{scope}
\begin{scope}
\path[clip] (  0.00,  0.00) rectangle (289.08,198.74);
\definecolor{drawColor}{RGB}{0,0,0}

\node[text=drawColor,anchor=base,inner sep=0pt, outer sep=0pt, scale=  1.00] at ( 41.49,  2.85) {2000};

\node[text=drawColor,anchor=base,inner sep=0pt, outer sep=0pt, scale=  1.00] at (107.89,  2.85) {2006};

\node[text=drawColor,anchor=base,inner sep=0pt, outer sep=0pt, scale=  1.00] at (174.30,  2.85) {2011};

\node[text=drawColor,anchor=base,inner sep=0pt, outer sep=0pt, scale=  1.00] at (240.70,  2.85) {2014};
\end{scope}
  \end{tikzpicture}}{Instituto Nacional de Estadística}{}
\cajita{Alfabetismo en jóvenes por área de residencia}{ 0}{Tasa de alfabetismo en población de 15 a 24 años por área de residencia}{República de Guatemala, Encovi 2014, en porcentaje}{\ \\[0mm]\begin{tikzpicture}[x=1pt,y=1pt]  \input{graficas/odm/25_14.tex}  \end{tikzpicture}}{Instituto Nacional de Estadística}{}
\cajota{Alfabetismo en jóvenes en los departamentos}{ 0}{Tasa de alfabetismo en población de 15 a 24 años por departamento}{República de Guatemala, Encovi 2014, en porcentaje}{\ \\[0mm]\begin{tikzpicture}[x=1pt,y=1pt]  \input{graficas/odm/25_14.tex}  \end{tikzpicture}}{Instituto Nacional de Estadística}{}
\cajita{Mujeres empleadas remuneradas en el sector no agrícola}{ Este indicador mide la igualdad de acceso al empleo remunerado, que condiciona integración en economía monetaria. Indica el grado de apertura de mujeres a los mercados de trabajo. 

Se puede observar, que la proporción de mujeres entre los empleados remunerados en el sector no agrícola, prácticamente no ha variado entre 2000 y 20014, ya que se ha mantenido en cerca del 44\% del total de empleados remunerados en el sector agrícola.
}{Proporción de mujeres entre los empleados remunerados en el sector no agrícola}{República de Guatemala, serie histórica por Encovi, en porcentaje}{\ \\[0mm]\begin{tikzpicture}[x=1pt,y=1pt]  % Created by tikzDevice version 0.8.1 on 2015-11-05 13:54:09
% !TEX encoding = UTF-8 Unicode
\definecolor{fillColor}{RGB}{255,255,255}
\path[use as bounding box,fill=fillColor,fill opacity=0.00] (0,0) rectangle (289.08,198.74);
\begin{scope}
\path[clip] (  0.00,  0.00) rectangle (289.08,198.74);

\path[] (  0.00,  0.00) rectangle (289.08,198.74);
\end{scope}
\begin{scope}
\path[clip] (  0.00,  0.00) rectangle (289.08,198.74);

\path[] (  1.64, 17.78) rectangle (280.54,191.48);

\path[] (  1.64, 43.30) --
	(280.54, 43.30);

\path[] (  1.64, 78.54) --
	(280.54, 78.54);

\path[] (  1.64,113.79) --
	(280.54,113.79);

\path[] (  1.64,149.04) --
	(280.54,149.04);

\path[] (  1.64,184.29) --
	(280.54,184.29);

\path[] (  1.64, 25.67) --
	(280.54, 25.67);

\path[] (  1.64, 60.92) --
	(280.54, 60.92);

\path[] (  1.64, 96.17) --
	(280.54, 96.17);

\path[] (  1.64,131.42) --
	(280.54,131.42);

\path[] (  1.64,166.67) --
	(280.54,166.67);

\path[] ( 41.49, 17.78) --
	( 41.49,191.48);

\path[] (107.89, 17.78) --
	(107.89,191.48);

\path[] (174.30, 17.78) --
	(174.30,191.48);

\path[] (240.70, 17.78) --
	(240.70,191.48);
\definecolor{drawColor}{RGB}{0,0,255}

\path[draw=drawColor,line width= 1.7pt,line join=round] ( 41.49,182.88) --
	(107.89,183.59) --
	(174.30,180.77) --
	(240.70,179.01);
\definecolor{drawColor}{RGB}{0,0,0}

\node[text=drawColor,anchor=base,inner sep=0pt, outer sep=0pt, scale=  1.01] at ( 41.49,171.01) {44.6};

\node[text=drawColor,anchor=base,inner sep=0pt, outer sep=0pt, scale=  1.01] at (107.89,187.54) {44.8};

\node[text=drawColor,anchor=base west,inner sep=0pt, outer sep=0pt, scale=  1.01] at (174.30,184.72) {44.0};

\node[text=drawColor,anchor=base,inner sep=0pt, outer sep=0pt, scale=  1.01] at (240.70,167.14) {43.5};

\path[draw=drawColor,line width= 0.1pt,line join=round] (  1.64, 25.67) -- (280.54, 25.67);

\path[] (  1.64, 17.78) rectangle (280.54,191.48);
\end{scope}
\begin{scope}
\path[clip] (  0.00,  0.00) rectangle (289.08,198.74);

\path[] (  1.64, 17.78) --
	(  1.64,191.48);
\end{scope}
\begin{scope}
\path[clip] (  0.00,  0.00) rectangle (289.08,198.74);

\path[] (  0.00, 25.67) --
	(  1.64, 25.67);

\path[] (  0.00, 60.92) --
	(  1.64, 60.92);

\path[] (  0.00, 96.17) --
	(  1.64, 96.17);

\path[] (  0.00,131.42) --
	(  1.64,131.42);

\path[] (  0.00,166.67) --
	(  1.64,166.67);
\end{scope}
\begin{scope}
\path[clip] (  0.00,  0.00) rectangle (289.08,198.74);

\path[] (  1.64, 17.78) --
	(280.54, 17.78);
\end{scope}
\begin{scope}
\path[clip] (  0.00,  0.00) rectangle (289.08,198.74);

\path[] ( 41.49, 13.51) --
	( 41.49, 17.78);

\path[] (107.89, 13.51) --
	(107.89, 17.78);

\path[] (174.30, 13.51) --
	(174.30, 17.78);

\path[] (240.70, 13.51) --
	(240.70, 17.78);
\end{scope}
\begin{scope}
\path[clip] (  0.00,  0.00) rectangle (289.08,198.74);
\definecolor{drawColor}{RGB}{0,0,0}

\node[text=drawColor,anchor=base,inner sep=0pt, outer sep=0pt, scale=  1.00] at ( 41.49,  2.85) {2000};

\node[text=drawColor,anchor=base,inner sep=0pt, outer sep=0pt, scale=  1.00] at (107.89,  2.85) {2006};

\node[text=drawColor,anchor=base,inner sep=0pt, outer sep=0pt, scale=  1.00] at (174.30,  2.85) {2011};

\node[text=drawColor,anchor=base,inner sep=0pt, outer sep=0pt, scale=  1.00] at (240.70,  2.85) {2014};
\end{scope}
  \end{tikzpicture}}{Instituto Nacional de Estadística}{}
\cajita{Mujeres empleadas remuneradas en el sector no agrícola por área de residencia}{ 0}{Proporción de mujeres entre los empleados remunerados en el sector no agrícola por área de residencia}{República de Guatemala, Encovi 2014, en porcentaje}{\ \\[0mm]\begin{tikzpicture}[x=1pt,y=1pt]  \input{graficas/odm/25_17.tex}  \end{tikzpicture}}{Instituto Nacional de Estadística}{}
\cajota{Mujeres empleadas remuneradas en el sector no agrícola en los departamentos}{ 0}{Proporción de mujeres entre los empleados remunerados en el sector no agrícola por departamento}{República de Guatemala, Encovi 2014, en porcentaje}{\ \\[0mm]\begin{tikzpicture}[x=1pt,y=1pt]  \input{graficas/odm/25_17.tex}  \end{tikzpicture}}{Instituto Nacional de Estadística}{}
\cajita{Partos con asistencia de personal de salud }{ 0}{Proporción de partos con asistencia de médico o ginecólogo}{República de Guatemala, serie histórica por Encovi, en porcentaje}{\ \\[0mm]\begin{tikzpicture}[x=1pt,y=1pt]  % Created by tikzDevice version 0.8.1 on 2015-11-05 13:54:13
% !TEX encoding = UTF-8 Unicode
\definecolor{fillColor}{RGB}{255,255,255}
\path[use as bounding box,fill=fillColor,fill opacity=0.00] (0,0) rectangle (289.08,198.74);
\begin{scope}
\path[clip] (  0.00,  0.00) rectangle (289.08,198.74);

\path[] (  0.00,  0.00) rectangle (289.08,198.74);
\end{scope}
\begin{scope}
\path[clip] (  0.00,  0.00) rectangle (289.08,198.74);

\path[] (  1.64, 17.78) rectangle (280.54,191.48);

\path[] (  1.64, 51.27) --
	(280.54, 51.27);

\path[] (  1.64,102.45) --
	(280.54,102.45);

\path[] (  1.64,153.64) --
	(280.54,153.64);

\path[] (  1.64, 25.67) --
	(280.54, 25.67);

\path[] (  1.64, 76.86) --
	(280.54, 76.86);

\path[] (  1.64,128.05) --
	(280.54,128.05);

\path[] (  1.64,179.24) --
	(280.54,179.24);

\path[] ( 41.49, 17.78) --
	( 41.49,191.48);

\path[] (107.89, 17.78) --
	(107.89,191.48);

\path[] (174.30, 17.78) --
	(174.30,191.48);

\path[] (240.70, 17.78) --
	(240.70,191.48);
\definecolor{drawColor}{RGB}{0,0,255}

\path[draw=drawColor,line width= 1.7pt,line join=round] ( 41.49,127.75) --
	(107.89,154.16) --
	(174.30,166.99) --
	(240.70,183.59);
\definecolor{drawColor}{RGB}{0,0,0}

\node[text=drawColor,anchor=base,inner sep=0pt, outer sep=0pt, scale=  1.01] at ( 41.49,115.88) {39.9};

\node[text=drawColor,anchor=base east,inner sep=0pt, outer sep=0pt, scale=  1.01] at (104.78,154.16) {50.2};

\node[text=drawColor,anchor=base east,inner sep=0pt, outer sep=0pt, scale=  1.01] at (171.18,166.99) {55.2};

\node[text=drawColor,anchor=base,inner sep=0pt, outer sep=0pt, scale=  1.01] at (240.70,187.54) {61.7};

\path[draw=drawColor,line width= 0.1pt,line join=round] (  1.64, 25.67) -- (280.54, 25.67);

\path[] (  1.64, 17.78) rectangle (280.54,191.48);
\end{scope}
\begin{scope}
\path[clip] (  0.00,  0.00) rectangle (289.08,198.74);

\path[] (  1.64, 17.78) --
	(  1.64,191.48);
\end{scope}
\begin{scope}
\path[clip] (  0.00,  0.00) rectangle (289.08,198.74);

\path[] (  0.00, 25.67) --
	(  1.64, 25.67);

\path[] (  0.00, 76.86) --
	(  1.64, 76.86);

\path[] (  0.00,128.05) --
	(  1.64,128.05);

\path[] (  0.00,179.24) --
	(  1.64,179.24);
\end{scope}
\begin{scope}
\path[clip] (  0.00,  0.00) rectangle (289.08,198.74);

\path[] (  1.64, 17.78) --
	(280.54, 17.78);
\end{scope}
\begin{scope}
\path[clip] (  0.00,  0.00) rectangle (289.08,198.74);

\path[] ( 41.49, 13.51) --
	( 41.49, 17.78);

\path[] (107.89, 13.51) --
	(107.89, 17.78);

\path[] (174.30, 13.51) --
	(174.30, 17.78);

\path[] (240.70, 13.51) --
	(240.70, 17.78);
\end{scope}
\begin{scope}
\path[clip] (  0.00,  0.00) rectangle (289.08,198.74);
\definecolor{drawColor}{RGB}{0,0,0}

\node[text=drawColor,anchor=base,inner sep=0pt, outer sep=0pt, scale=  1.00] at ( 41.49,  2.85) {2000};

\node[text=drawColor,anchor=base,inner sep=0pt, outer sep=0pt, scale=  1.00] at (107.89,  2.85) {2006};

\node[text=drawColor,anchor=base,inner sep=0pt, outer sep=0pt, scale=  1.00] at (174.30,  2.85) {2011};

\node[text=drawColor,anchor=base,inner sep=0pt, outer sep=0pt, scale=  1.00] at (240.70,  2.85) {2014};
\end{scope}
  \end{tikzpicture}}{Instituto Nacional de Estadística}{}
\cajita{Partos con asistencia de personal de salud por área de residencia}{ 0}{Proporción de partos con asistencia de médico o ginecólogo por área de residencia}{República de Guatemala, Encovi 2014, en porcentaje}{\ \\[0mm]\begin{tikzpicture}[x=1pt,y=1pt]  \input{graficas/odm/25_40.tex}  \end{tikzpicture}}{Instituto Nacional de Estadística}{}
\cajota{Partos con asistencia de personal de salud en los departamentos}{ 0}{Proporción de partos con asistencia de médico o ginecólogo por departamento}{República de Guatemala, Encovi 2014, en porcentaje}{\ \\[0mm]\begin{tikzpicture}[x=1pt,y=1pt]  \input{graficas/odm/25_40.tex}  \end{tikzpicture}}{Instituto Nacional de Estadística}{}
\cajita{Atención del parto en centros públicos}{ 0}{Proporción de partos atendidos en hospital, centro o puesto de salud público}{República de Guatemala, serie histórica por Encovi, en porcentaje}{\ \\[0mm]\begin{tikzpicture}[x=1pt,y=1pt]  \input{graficas/odm/25_42.tex}  \end{tikzpicture}}{Instituto Nacional de Estadística}{}
\cajita{Atención del parto en centros públicos por área de residencia}{ 0}{Proporción de partos atendidos en hospital, centro o puesto de salud público por área de residencia}{República de Guatemala, Encovi 2014, en porcentaje}{\ \\[0mm]\begin{tikzpicture}[x=1pt,y=1pt]  \input{graficas/odm/25_43.tex}  \end{tikzpicture}}{Instituto Nacional de Estadística}{}
\cajota{Atención del parto en centros públicos en los departamentos}{ 0}{Proporción de partos atendidos en hospital, centro o puesto de salud público por departamento }{República de Guatemala, Encovi 2014, en porcentaje}{\ \\[0mm]\begin{tikzpicture}[x=1pt,y=1pt]  \input{graficas/odm/25_43.tex}  \end{tikzpicture}}{Instituto Nacional de Estadística}{}
\cajita{Acceso a agua mejorada}{ 0}{Proporción de la población con acceso a fuentes mejoradas de abastecimiento de agua potable}{República de Guatemala, serie histórica por Encovi, en porcentaje}{\ \\[0mm]\begin{tikzpicture}[x=1pt,y=1pt]  \input{graficas/odm/25_45.tex}  \end{tikzpicture}}{Instituto Nacional de Estadística}{}
\cajita{Acceso a agua mejorada por área de residencia}{ 0}{Proporción de la población con acceso a fuentes mejoradas de abastecimiento de agua potable según área de residencia}{República de Guatemala, Encovi 2014, en porcentaje}{\ \\[0mm]\begin{tikzpicture}[x=1pt,y=1pt]  \input{graficas/odm/25_46.tex}  \end{tikzpicture}}{Instituto Nacional de Estadística}{}
\cajota{Acceso a agua mejorada en los departamentos}{ 0}{Proporción de la población con acceso a fuentes mejoradas de abastecimiento de agua potable por departamento}{República de Guatemala, Encovi 2014, en porcentaje}{\ \\[0mm]\begin{tikzpicture}[x=1pt,y=1pt]  \input{graficas/odm/25_46.tex}  \end{tikzpicture}}{Instituto Nacional de Estadística}{}
\cajita{Acceso a saneamiento mejorado}{ 0}{Proporción de la población con acceso a servicios de saneamiento mejorados}{República de Guatemala, serie histórica por Encovi, en porcentaje}{\ \\[0mm]\begin{tikzpicture}[x=1pt,y=1pt]  % Created by tikzDevice version 0.8.1 on 2015-11-05 13:54:24
% !TEX encoding = UTF-8 Unicode
\definecolor{fillColor}{RGB}{255,255,255}
\path[use as bounding box,fill=fillColor,fill opacity=0.00] (0,0) rectangle (289.08,198.74);
\begin{scope}
\path[clip] (  0.00,  0.00) rectangle (289.08,198.74);

\path[] (  0.00,  0.00) rectangle (289.08,198.74);
\end{scope}
\begin{scope}
\path[clip] (  0.00,  0.00) rectangle (289.08,198.74);

\path[] (  1.64, 17.78) rectangle (280.54,191.48);

\path[] (  1.64, 52.76) --
	(280.54, 52.76);

\path[] (  1.64,106.95) --
	(280.54,106.95);

\path[] (  1.64,161.14) --
	(280.54,161.14);

\path[] (  1.64, 25.67) --
	(280.54, 25.67);

\path[] (  1.64, 79.86) --
	(280.54, 79.86);

\path[] (  1.64,134.04) --
	(280.54,134.04);

\path[] (  1.64,188.23) --
	(280.54,188.23);

\path[] ( 41.49, 17.78) --
	( 41.49,191.48);

\path[] (107.89, 17.78) --
	(107.89,191.48);

\path[] (174.30, 17.78) --
	(174.30,191.48);

\path[] (240.70, 17.78) --
	(240.70,191.48);
\definecolor{drawColor}{RGB}{0,0,255}

\path[draw=drawColor,line width= 1.7pt,line join=round] ( 41.49,145.42) --
	(107.89,173.33) --
	(174.30,177.39) --
	(240.70,183.59);
\definecolor{drawColor}{RGB}{0,0,0}

\node[text=drawColor,anchor=base,inner sep=0pt, outer sep=0pt, scale=  1.01] at ( 41.49,133.55) {44.2};

\node[text=drawColor,anchor=base east,inner sep=0pt, outer sep=0pt, scale=  1.01] at (104.78,173.33) {54.5};

\node[text=drawColor,anchor=base east,inner sep=0pt, outer sep=0pt, scale=  1.01] at (171.18,177.39) {56.0};

\node[text=drawColor,anchor=base,inner sep=0pt, outer sep=0pt, scale=  1.01] at (240.70,187.54) {58.3};

\path[draw=drawColor,line width= 0.1pt,line join=round] (  1.64, 25.67) -- (280.54, 25.67);

\path[] (  1.64, 17.78) rectangle (280.54,191.48);
\end{scope}
\begin{scope}
\path[clip] (  0.00,  0.00) rectangle (289.08,198.74);

\path[] (  1.64, 17.78) --
	(  1.64,191.48);
\end{scope}
\begin{scope}
\path[clip] (  0.00,  0.00) rectangle (289.08,198.74);

\path[] (  0.00, 25.67) --
	(  1.64, 25.67);

\path[] (  0.00, 79.86) --
	(  1.64, 79.86);

\path[] (  0.00,134.04) --
	(  1.64,134.04);

\path[] (  0.00,188.23) --
	(  1.64,188.23);
\end{scope}
\begin{scope}
\path[clip] (  0.00,  0.00) rectangle (289.08,198.74);

\path[] (  1.64, 17.78) --
	(280.54, 17.78);
\end{scope}
\begin{scope}
\path[clip] (  0.00,  0.00) rectangle (289.08,198.74);

\path[] ( 41.49, 13.51) --
	( 41.49, 17.78);

\path[] (107.89, 13.51) --
	(107.89, 17.78);

\path[] (174.30, 13.51) --
	(174.30, 17.78);

\path[] (240.70, 13.51) --
	(240.70, 17.78);
\end{scope}
\begin{scope}
\path[clip] (  0.00,  0.00) rectangle (289.08,198.74);
\definecolor{drawColor}{RGB}{0,0,0}

\node[text=drawColor,anchor=base,inner sep=0pt, outer sep=0pt, scale=  1.00] at ( 41.49,  2.85) {2000};

\node[text=drawColor,anchor=base,inner sep=0pt, outer sep=0pt, scale=  1.00] at (107.89,  2.85) {2006};

\node[text=drawColor,anchor=base,inner sep=0pt, outer sep=0pt, scale=  1.00] at (174.30,  2.85) {2011};

\node[text=drawColor,anchor=base,inner sep=0pt, outer sep=0pt, scale=  1.00] at (240.70,  2.85) {2014};
\end{scope}
  \end{tikzpicture}}{Instituto Nacional de Estadística}{}
\cajita{Acceso a saneamiento mejorado por área de residencia}{ 0}{Proporción de la población con acceso a servicios de saneamiento mejorados por área de residencia}{República de Guatemala, Encovi 2014, en porcentaje}{\ \\[0mm]\begin{tikzpicture}[x=1pt,y=1pt]  \input{graficas/odm/25_49.tex}  \end{tikzpicture}}{Instituto Nacional de Estadística}{}
\cajota{Acceso a saneamiento mejorado en los departamentos}{ 0}{Proporción de la población con acceso a servicios de saneamiento mejorados por departamento}{República de Guatemala, Encovi 2014, en porcentaje}{\ \\[0mm]\begin{tikzpicture}[x=1pt,y=1pt]  \input{graficas/odm/25_49.tex}  \end{tikzpicture}}{Instituto Nacional de Estadística}{}


