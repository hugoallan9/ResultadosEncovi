 Para el año 2000, el índice de Atkinson\footnote{El índice de Atkinson mide la desigualdad en términos de la pérdida de bienestar social, debido a la dispersión de los ingresos, donde $\varepsilon$ se interpreta como un parámetro de aversión a la desigualdad.\\\\ 
	Para $\varepsilon = \mbox{1} $ el índice se calcula con la fórmula: 
	\[ A =  1 - \prod_{i=1}^{N}\left(\frac{x_i}{\mu}
\right)^{\frac{1}{N}}, \] donde $N$ es la población total, $\mu$ es la media del ingreso y $x_i$ es el ingreso de la persona $i$.} con el parámetro de aversión a la desigualdad igual a uno, era de 0.52. Al comparar entre 2006 y 2014, se observa una pequeña reducción en el índice de 0.45 a 0.41, respectivamente.