Para el año 2000, el 15.7\% de la población se encontraba por debajo de la línea de pobreza extrema\footnote{El indicador para medir la pobreza extrema es el FGT (Foster, Greer y Thorbecke) que se calcula mediante la fórmula: \[ FGT(0)  = \frac{H}{N}\times 100,  \] donde $H$  es la cantidad de casos que reportan ingresos menores o iguales a la línea de pobreza extrema y $N$ es el la población total.}. 

Entre 2000 y 2006  el nivel de pobreza extrema se mantuvo ya que aumentó en menos de un punto porcentual; mientras que para 2014, hubo un aumento de la pobreza extrema de 8.1 puntos porcentuales.