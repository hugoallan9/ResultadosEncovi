La proporción de la población que trabaja por cuenta propia, agrícola y no agrícola, es mayor en el área rural que en el área urbana, la diferencia es alrededor de cinco puntos porcentuales. Es decir, que en el área rural es mayor la proporción de ocupados con un empleo no asalariado\footnote{Se considera a un trabajador asalariado como aquel que trabaja para un patrón, empresa o negocio, institución o dependencia, regidos por un contrato escrito o de palabra a cambio de un jornal, sueldo o salario.} que en el  área urbana. 