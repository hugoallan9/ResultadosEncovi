Para 2014, el 59.3\% de la población se encontraba en pobreza\footnote{El indicador para medir pobreza es el FGT (Foster, Greer y Thorbecke) que se calcula mediante la fórmula: \[ FGT(0)  = \frac{H}{N}\times 100,  \] donde $H$  es la cantidad de casos que reportan un consumo  menor o igual a la línea de pobreza total y $N$ es el la población total.}, es decir, más de la mitad de la población tenía un consumo por debajo de Q10,218 al año. 

Se puede observar en la gráfica  que entre 2000 y 2014, la pobreza total aumentó en 2.9 puntos porcentuales, pasando de 56.4\% en 2000 a 59.3\% en 2014.