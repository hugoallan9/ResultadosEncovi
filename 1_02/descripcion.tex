Trae fórmula Nota: indicador FGT (Foster, Greer y Thorbecke) Para 2014, el 59.3\% de la población se encontraba por debajo de la línea de pobreza total, es decir, más de la mitad de la población tenía un consumo por debajo de Q10,218 al año. Se puede observar en la gráfica  que entre 2000 y 2014, la pobreza total aumentó en 2.9 puntos porcentuales, pasando de 56.4\% en 2000 a 59.3\% en 2014.