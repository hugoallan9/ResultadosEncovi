\pagestyle{soloarriba}
\clearpage

$\ $
\vspace{14.5cm}

\noindent\begin{tabular}{p{0.9cm}p{6.8cm}}
& 2015.$\,$ Guatemala, Centro América \\
&\Bold Instituto Nacional de Estadística\\[-0.4cm]
&\color{blue!50!black}\url{www.ine.gob.gt}\\[0.9cm]
\end{tabular}\\
\noindent\begin{tabular}{p{0.9cm}p{6.8cm}}
& Está permitida la reproducción parcial o total de los contenidos de este documento con la mención de la fuente. \\[0.5cm]
 
& Este documento fue elaborado empleando  {\Sans R}, Inkscape y {\Logos \XeLaTeX}.\\
\end{tabular} 


\clearpage



	
	\clearpage
	\newpage $\ $

$\ $
\vspace{0.0cm}

\begin{center}
	{\Bold \LARGE AUTORIDADES}\\[1cm]
	
	
	{\Bold \large \color{color1!89!black} JUNTA  DIRECTIVA} \\[0.4cm]
	
	{ \Bold Ministerio de Economía}  		\\ 
	Titular: Jorge Méndez Herbruger   \\ 
	Suplente: Jacobo Rey Sigfrido Lee Leiva  \\ [0.4cm] 
	
	{\Bold Ministerio de Finanzas} \\ 
	Titular: Dorval José Carías Samayoa \\ 
	Suplente: Edwin Oswaldo Martínez Cameros\\[0.4cm] 
	
	{\Bold Ministerio de Agricultura, Ganadería y Alimentación} \\ 
	Titular: José Sebastian Marcucci Ruíz   \\ 
	Suplente: Henry Giovanni Vásquez Kilkan \\ [0.4cm] 
	
	{\Bold Ministerio de Energía y Minas}\\ 
	Titular: Juan Pablo Ligorría Arroyo \\ 
	Suplente: Jorge David Calvo Drago\\ [0.4cm]
	{\Bold Secretaría de Planificación y Programación de la Presidencia}   \\
	Titular: Ekaterina Arbolievna Parrilla Artuguina   \\ 
	Suplente: Dora Marina Coc Yup\\ [0.4cm] 
	{\Bold Banco de Guatemala} \\ 
	Titular: Julio Roberto Suárez Guerra \\ 
	Suplente: Sergio Francisco Recinos Rivera\\ [0.4cm] 
	{\Bold Universidad de San Carlos de Guatemala de Guatemala} \\ 
	Titular: Murphy Olimpo Paiz Recinos   \\
	Suplente: Oscar René Paniagua Carrera  \\ [0.4cm]
	{\Bold Universidades Privadas} \\
	Titular: Miguel Ángel Franco de León \\			 Suplente: Ariel Rivera Irías\\ [0.4cm] 
	{\Bold Comité Coordinador de \ Asociaciones  Agrícolas, Comerciales,Industriales y Financieras} \\ 
	Titular: Juan Raúl Aguilar Kaehler \\
	Suplente:  Oscar Augusto Sequeira García  \\ [0.4cm]
	
	{\Bold \large \color{color1!89!black} GERENCIA}\\[0.2cm]
	Gerente:  Rubén Darío Narciso Cruz		\\
	Subgerente Técnico:  Jaime Roberto Mejía Salguero\\
	Subgerente Administrativo Financiero:  Orlando Roberto Monzón Girón\\ 
\end{center}
\clearpage

$\ $
\vspace{1cm}

\begin{center}
	{\Bold \LARGE EQUIPO RESPONSABLE}\\[2cm]
	
	{\Bold \large \color{color1!89!black} REVISIÓN GENERAL}\\[0.2cm]
	Rubén Darío Narciso Cruz\\[0.8cm]
	
	
	{\Bold \large \color{color1!89!black} EQUIPO TÉCNICO}\\[0.2cm]
	Jaime Mejía\\
	Carlos Mancia\\
	Carlos Ortiz\\
	Marvin Reyes\\
	Hugo Rivas\\
	Nelson Santacruz \\
	Luis Fernando Bonilla\\
	Pamela Escobar\\
	Vivian Guzmán\\
	Patricia Hernández\\
	Sucely Donis \\
	Fabiola Ramírez \\
	Hugo García \\
	Mynor Flores \\[0.8cm]
	
{\Bold \large \color{color1!89!black} DIAGRAMACIÓN Y DISEÑO}\\[0.2cm]
Ligia Morales\\
José Carlos Bonilla Aldana
\end{center}
\vfill
\hrule 
El Instituto Nacional de Estadística agradece el apoyo técnico brindado por el Banco Mundial para la realización de esta encuesta; en especial a Carlos Sobrado,Mario Navarrete y  Elizaveta Perova por su valiosa contribución. 
\cleardoublepage
$\ $\\[2cm]
\noindent {\Bold \huge Presentación}
\\\\
 
 La Encuesta Nacional de Condiciones de Vida \textemdash Encovi\textemdash, tiene como principal objetivo, conocer y evaluar las condiciones de vida de la población, así como determinar los niveles de pobreza existentes en Guatemala y los factores que los determinan.
 
 La Encovi adopta la metodología de las encuestas de condiciones de vida, que en lo fundamental, combinan aspectos cuantitativos y cualitativos mediante la aplicación de un conjunto integrado de formularios sobre la calidad de vida de los hogares y las personas. Esta perspectiva permite una mejor aproximación a los diferentes aspectos y componentes de la pobreza, es decir, a su carácter multidimensional. Permite además, abordar el estudio de la desigualdad y la identificación de mecanismos de intervención eficaz que promuevan mejoras sustantivas de las condiciones de vida.

Debido a la amplitud de los temas investigados en la Encovi, el informe de resultados de esta encuesta se presentará en tres tomos en el 2016. Sin embargo para que la población en general tenga oportunamente    los principales resultados de la Encovi 2014, se presenta este informe el cúal es un extracto del documento de mayor volumen que se publicará el próximo año.  El presente documento tiene tres capítulos: Pobreza, Desigualdad y Objetivos de Desarrollo del Milenio, los que permiten visualizar la evolución de las condiciones de vida del país en los últimos quince años. 


El Instituto Nacional de Estadística agradece a todas las instituciones públicas, privadas y de cooperación internacional que apoyaron este esfuerzo. Asimismo, a los hogares que abrieron sus puertas y  brindaron  la información  solicitada, sin la cual no hubiera sido posible  la elaboración de este informe.\\[1cm] 


\begin{center}
	\textbf{Rubén Darío Narciso Cruz}\\
	Gerente\\
	Instituto Nacional de Estadística
\end{center}

\cleardoublepage
